\documentclass[12pt, a4paper]{article}
\usepackage[legalpaper, margin=1.5in]{geometry}
\usepackage[T2A]{fontenc}
\usepackage[utf8]{inputenc}
\usepackage[english,russian]{babel}
\usepackage{mathrsfs}
\usepackage{amsfonts}

\begin{document}

12. \\ \\
Задача: \\ \\
Доказать, что можно проверить выполнимость хорновской формулы за 
полиномиальное время. \\ \\
Решение:

\begin{enumerate}
    \item Составим $U$ - множество всех дизъюнктов, состоящих из одной переменной.
    \item Если для какой-то переменной $x_i$, входящей в исходную хорновскую формулу
        верно: $x_i \in U ~ \& ~ \neg x_i \in U $, следовательно исходная формула не выполнима.
    \item Подставим вместо переменных, входящих в дизъюнкты из $U$, значения, обращающие 
        соответствующие дизъюнкты из $U$ в значение "Истина". Заметим, что новая формула, 
        получившаяся в результате подстановки, также будет хорновской (Действительно, если 
        дизъюнкт состоял из двух переменных, в результате подстановки он перейдёт в дизъюнкт из
        одной переменной, что в любом случае допустимо для хорновской формулы. Если же дизъюнкт
        состоял из $n$ переменных, то он перейдёт в дизъюнкт из $n-1$ переменных, и в нём по прежнему
        будет не более одной переменной без отрицания, так как подстановка не увеличивает число 
        переменных без отрицания. Также стоит отметить, что если дизъюнкт из исходной формулы не 
        содержал переменных, которые входят в $U$, то он остаётся неизменным.). В результате
        такой подстановки можем получить новую формулу, в которой появились новые дизъюнкты, 
        состоящие из 1 переменной. В таком случае, повторим шаги 1-3.
    \item В результате перейдём к хорновской формуле, состоящей только из дизъюнтков,
        которые содержат минимум 2 переменные, и, по определению, хотя бы одну с отрицанием.
        В таком случае, итоговая формула выполняется подстановкой вместо всех оставшихся переменных
        нулей. Также мы можем получить в результате подстановок просто истинное или ложное
        значение, тогда мы также легко можем судить о выполнимости исходной формулы: если 
        получили истинное значение - исходная формула выполняется, иначе - нет.

\end{enumerate}

Таким образом, при очень грубой оценке, время на составление множества $U$ и проверка из пункта
 2 не превосходит $O(m)$,
число повторений шагов 1-3 в худшем случае равно m, то время работы описанного алгоритма
можно оценить сверху как $O(m^2)$, где $m$ - 
число переменных, входящих в состав исходной формулы. А значит мы можем проверить выполнимость
хорновской формулы за полиномиальное время.

\end{document}

\documentclass[12pt]{article}

\usepackage[T2A]{fontenc}
\usepackage[utf8]{inputenc}
\usepackage[english,russian]{babel}
\usepackage{mathrsfs}
\usepackage{amsfonts}

\begin{document}

11. 

Задача: Доказать полиномиальную эквивалентность задач о клике (К),
вершинном покрытии (ВП) и независимом множестве (НМ) в графе.

Решение:

\begin{enumerate}
    \item К $\propto$ НМ, НМ $\propto$ К

        В $G ~\exists$ множество клика W размера не менее B $\iff$ 
        в $\overline{G} ~\exists$ НМ размера не менее B, где
        $\overline{G}$ - Граф: $V(\overline{G}) = V(G), E(\overline{G})~$ 
        - все рёбра, которых нет в G и только они.

        Понятно, что $\forall ~ u,v \in W: uv \in E(G)$, тогда для той же
        пары u и v $uv \notin E(\overline{G})$, по построению.

    \item ВП $\propto$ НМ, НМ $\propto$ ВП

        В $G~\exists$ НМ W размера не менее B $\iff$
        в $G~\exists$ ВП размера не менее $n-B, n=v(G)$ 

        Если W - НМ $\Rightarrow \forall ~ u,v \in HM, uv \notin E(G) ~ \Rightarrow$
        все рёбра покрыты оставшимися вершинами из 
        $ V(G) \backslash W \iff ~ \forall ~ e \in E(G) ~ \exists u \in V(G) \backslash W:$
        $u \in V(e)$

        Аналогично, если в $G~ \exists ~ $ ВП W размера не менее B $\Rightarrow$ в $G ~ \exists ~$ НМ
        размера не менее $n-B$
        Таким образом между любыми двумя вершинами из $V(G) \backslash W$ нет ребра, иначе хотя бы
        одна из них входила в вершинное покрытие.

    \item К $\propto$ ВП, ВП $\propto$ К

        По транзитивности:

        К $\propto$ НМ, НМ $\propto$ ВП $\Rightarrow$ К $\propto$ ВП

        ВП $\propto$ НМ, НМ $\propto$ К $\Rightarrow$ ВП $\propto$ К
\end{enumerate}

Таким образом, все эти задачи полиномиально эквивалентны.


\end{document}

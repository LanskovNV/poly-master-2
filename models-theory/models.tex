
\documentclass[12pt, a4paper]{article}
\usepackage[legalpaper, margin=1.5in]{geometry}
\usepackage[T2A]{fontenc}
\usepackage[utf8]{inputenc}
\usepackage[english,russian]{babel}
\usepackage{mathrsfs}
\usepackage{amsfonts}

\begin{document}

%******************************************************************
%******************************************************************
\begin{titlepage}
	\center
		Санкт-Петербургский Политехнический
		университет \\ Петра Великого\\
		Институт прикладной математики и механики
		\\ \textbf{Высшая школа прикладной математики и вычислительной физики}

	\vfill ~
	\textbf{
		\\ \large Реферат
	}
	\\	на тему
	\\ "Алгебраическая теория моделей"
	\\ по дисциплине
	\\ "Информационное обеспечение компьютерных систем"

	\vfill ~

	Выполнил студент гр. \textbf{3640102/00201} \\
	\textbf{Лансков.Н.В.} \\

\vfill

{\large}	Санкт-Петербург
\\ 2021
\end{titlepage}

\section{Теория моделей}

\textbf{Теория моделей} – раздел математической логики, изучающий модели формальных теорий,
соотношения между моделями и теориями и преобразования моделей. Предшественниками теории
моделей были Б.Больцано и Э.Шрёдер, осознавшие понятие выполнимости формулы на
интерпретации. В настоящий момент теория моделей делится на следующие разделы: классическая теория моделей, Алгебраическая теория моделей и интерпретации реализуемости.

\section{Классическая теория моделей}

Классическая теория моделей (КМТ), изучающая теоретико-множественные модели классических теорий.

КМТ берет начало от работ Лёвенгейма (1915) и Скулема (1920), установивших существование
моделей любой бесконечной мощности для любой непротиворечивой теории, имеющей бесконечную
модель. Этот результат вначале рассматривался как парадоксальный, потому что из него следовало
существование счетных моделей несчетных множеств, а мощность множества в те времена
содержательно интерпретировали как число элементов, по аналогии с конечными множествами, а не
как сложность его задания, как сейчас делается по аналогии с теорией алгоритмов. Фундаментальным
результатом КМТ явилась теорема Гёделя о полноте классической логики предикатов (первого
порядка), из которой следует существование моделей у любых (основанных на этой логике)
непротиворечивых теорий. В 70-е гг. выяснилось, что теорема Гёделя о полноте эквивалентна аксиоме
выбора множеств теории.
Если задана некоторая сигнатура (перечисление констант, функциональных символов и предикатов
вместе с числом аргументов у них), то (классической) интерпретацией данной сигнатуры является
непустое множество объектов – универсум интерпретации, и функция вычисления значения $\zeta$
сопоставляющая каждой константе – элемент универсума, n-местной функции f – функционал Un→U,
n-местному предикату Ρ – функционал $U_n \rightarrow {0,1}$. В интерпретации естественно определяется
понятие значения любого терма и любой формулы теории (точное определение истинности формулы в
интерпретации было впервые дано А.Тарским). Интерпретация называется моделью теории, если в ней
истинны все аксиомы теории. Еще одной формулировкой теоремы полноты Гёделя является
совпадение множества теорем с множеством формул, истинных в любой модели теории.
По теореме Мальцева о компактности, теория имеет модель тогда, и только тогда, когда любое
конечное число ее аксиом имеет модель. Эта теорема послужила основой для построения
нестандартных моделей традиционных математических объектов, таких, как действительные и
натуральные числа.
В самом деле, взяв в качестве теории все истинные на стандартной модели формулы и добавив новое
число ω и бесконечную совокупность аксиом ω>0, ω>1, ω >n, мы получаем, что любая конечная
совокупность новых аксиом удовлетворяется на стандартной модели. Значит, есть и модель, где они
все выполнены. Она сохраняет все выразимые на языке логики предикатов свойства стандартной
модели, но пополнена новыми элементами.
Позитивно использовал существование нестандартных моделей А.Робинсон (1960). Он показал, что в
нестандартной модели анализа можно на строгой основе возродить методы математиков 17–18 вв.,
использовавших бесконечно малые и бесконечно большие величины. Основополагающим явился здесь
результат, что любое конечное нестандартное число однозначно разлагается в сумму стандартного и
бесконечно малого. Далее, сохранение всех выразимых свойств используется для установления
принципов переноса, которые позволяют отбрасывать бесконечно малые либо доказывать общее
утверждение о стандартных числах на основе рассмотрения одной бесконечно малой либо бесконечно
большой величины. Но здесь приходится строго разделять формулы стандартного языка и формулы
метаязыка, говорящего о нестандартной модели. В частности, утверждения, явно включающие
предикат «быть (не)стандартным», уже могут нарушать все свойства стандартной модели. Дальнейшее
развитие нестандартного анализа привело к теории полумножеств Г.Хаека и к альтернативной теории
множеств С.Вопенки, где конечные нестандартные совокупности могут включать бесконечные
подклассы.
Современная КМТ развивается во многих направлениях, большинство из которых в данный момент
имеют дело со сложнейшими идеальными математическими понятиями (абстрактными объектами)
без выхода на общенаучные либо методологические результаты. Правда, приятным исключением
является совокупность теорем, характеризующих теории частного вида через их модели. ∀-теория –
это теория, все аксиомы которой имеют вид ∀ А( ), где x – совокупность переменных, и А не
содержит кванторов.
Теорема Лося. Теория представима как ∀-теория тогда, и только тогда, когда каждая подсистема ее
модели также является ее моделью.
Эта теорема при внешней простоте формулировки требует использования абстрактных и сложных
конструкций КМТ. Таковы же и другие теоремы характеризации. В частности, совокупность систем
называется многообразием, если она является множеством моделей теории с аксиомами вида ∀ P(t(
)), где Р – предикат. Многообразия – это ∀-теории, модели которых сохраняются при гомоморфизмах.
Теоремы характеризации используются в современной информатике для описания абстрактных типов
данных.

\section{Алгебраическая теория моделей}

Алгебраическая теория моделей (ATM), изучающая прежде всего модели неклассических логик,
базирующиеся на обобщенной семантике истинностных значений. Теория моделей Крипке (СВМ),
изучающая модели неклассических логик, базирующиеся на возможных миров семантике.

ATM началась с предложенной Линденбаумом и Тарским концепции, согласно которой любая теория
может рассматриваться как алгебра, операциями которой являются логические связки, а объектами –
классы формул, для которых доказуема эквивалентность. Такая алгебра называется алгеброй
Линденбаума-Тарского (ЛТ-алгеброй) теории. ЛТ-алгебра классической теории – булева алгебра. ЛТ-
алгебра интуиционистской – псевдобулева, теории в модальной логике S4 – булева алгебра с
замыканиями. Данный подход был вторым основанием и инструментом для построения альтернативной
теории множеств. Для неклассических логик он математически эквивалентен СВМ и поэтому в
последнее время употребляется менее интенсивно. Трудностью в ATM является интерпретация
кванторов. Для данной цели была развита теория цилиндрических алгебр.
Семантика возможных миров (СВМ) предлагалась уже Аристотелем, который рассматривал теорию
модальных суждений. Ее предшественником можно считать Г.Лейбница, который явно ввел понятие
возможного мира. В современном виде она впервые была предложена для частного случая
интуиционистской логики Э.Бетом (1954) и последовательно развита для целого ряда логик
С.Крипке, имя которого она и получила.
При СВМ интерпретациях имеется некоторая алгебраическая система классических (либо, в более
тонких случаях, алгебраических) моделей, называемых мирами, связанных отношениями и порою
функциями. Для модальных логик СВМ интерпретации обычно используют единственное бинарное
отношение достижимости.
Логика L называется шкальной, если любая интерпретация с той же системой миров, что у модели L,
также является моделью L. Т.о., шкальные логики накладывают ограничения не на отдельные миры, а
на их внешние взаимосвязи.
Один из интереснейших результатов современной СВМ – перечисление всех суперинтуиционистских и
модальных пропозициональных логик, обладающих интерполяционным свойством Крейга: для любой
доказуемой импликации А ⇒ В найдется формула С, содержащая лишь термины, общие для А и В,
такая, что доказуемы А ⇒ С и С ⇒ В. В работах Л.Л.Максимовой показано, что логик, обладающих
свойством Крейга, конечное число.
Математическая структура вынуждения, использованная П.Дж.Коэном как промежуточный шаг для
построения нестандартных классических моделей теоретико-множественных систем, позднее получила
название моделей Крипке для интуиционистской логики. С их помощью решена проблема
Гильберта: доказана независимость аксиомы выбора и континуум-гипотезы. Далее, теми же методами
установлена невозможность явного построения, в частности, неизмеримого множества действительных
чисел и нестандартной модели анализа. Исторически это было одно из первых использований СВМ.

\section{Интерпретации реализуемоости}

Моделирующие логики и теории как исчисления задач.

Последний класс моделей – ИР Колмогоровская интерпретация допускает значительную гибкость в
классе используемых функционалов, поэтому в ИР используются и алгоритмы, и топологические
пространства с непрерывными преобразованиями, и категории, и формальные выводы, и комбинации
данных объектов.
Наиболее значительные в методологических аспектах результаты, полученные при помощи ИР за
последнее время, следующие. Доказана совместимость с интуиционистской математикой моделей
брарровских концепций творящего субъекта и беззаконных последовательностей (см. Интуиционизм)
и построены модели вычислимости, основанные на данных концепциях. Т.о., обосновано, что
содержательный вычислительный метод может быть представлен как композиция алгоритма,
творческого процесса и физических измерений. Доказано, что для многих аксиоматических систем
добавление аксиомы выбора к конструктивному анализу и к теории множеств с интуиционистской
логикой не нарушает эффективности доказательств. Т.о., аксиома выбора на самом деле не приводит
сама по себе к чистым теоремам существования; в данном смысле она концептуально противоречит
исключенного третьего закону, который с необходимостью приводит к таким теоремам.

\end{document}
